%!TEX root = ../main.tex

% \section{Section Title}
% \subsection{Subsection}
% \subsubsection{Subsubsection}
% \paragraph{Paragraph}
% \subparagraph{Subparagraph}

% \ul = underline
% \st = strikethrough
% \hl = highlight
% \textbf = bold face
% \textit = italic face
% \textsl = slanted
% \textsf = sans serif

%%%%%%%%%%%%%%%%%%%%%%%%%%%%%%%%%%%%%%%%%%%%%%%%%%%%%%%%%%%%%%%%%%%%%%%%%

This chapter provides an introduction to this thesis.
In \cref{s:introduction-motivation} the general motivation is discussed.
The section is followed by outlining the research goals in \cref{s:introduction-researchgoals}.
\Cref{s:introduction-researchmethodology} gives an insight into research methods used in order to answer the raised research questions.
Finally, \cref{s:introduction-structureofthisthesis} gives an outlook of the structure of the remaining thesis.

\section{Motivation}
\label{s:introduction-motivation}

Many studies have been published which are trying to predict the stock market movement.
As the \ac{EMH} states that financial market movements depend on news, current events and product releases and all these factors will have significant impact on a company's stock value
\cite{fama1965behavior}.
Due the fact that news and current events are unpredictable stock market prices are following a random walk pattern and cannot predicted with more than 50\% accuracy
\cite{Pagolu2016a}.

Many internet users are microblogging nowadays.
Millions of messages are published daily on popular websites which provides microblogging services, such as Twitter, Tumblr and Facebook.
These published messages describing the personal life, opinions or current issues.
The more users are post about products and services they use the more microblogging websites become a valuable source of peoples opinions and sentiments.
Therefore, this data can be used for marketing, social studies and as a measure of public opinion
\cite{Patodkar2016a, Pagolu2016a}.

As most Twitter messages have a maximum length of 140 characters and speaks public opinion on a topic precisely
\cite{Pagolu2016a}.

\section{Research Goals}
\label{s:introduction-researchgoals}

According to the factors presented in \cref{s:introduction-motivation} the central research question can be formulated:
\emph{Are the stock market movements dependent from public opinion extracted from Twitter?}

The goal of this research to analyze the correlation between sentiment of tweets and share movement of automotive companies.
This goal will be met by achieving the following objectives:

\begin{itemize}
    \item \textbf{G1} - Determine companies, keywords and stock symbols to analyze
    \item \textbf{G2} - Gather tweets and their sentiments and stock prices
    \item \textbf{G3} - Comparing sentiment time series with share prices
\end{itemize}

From definitions of goals and having the central question in mind the following research questions are set up:



\section{Research Methodology}
\label{s:introduction-researchmethodology}

\section{Structure of this Thesis}
\label{s:introduction-structureofthisthesis}
