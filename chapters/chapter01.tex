% \section{Section Title}
% \subsection{Subsection}
% \subsubsection{Subsubsection}
% \paragraph{Paragraph}
% \subparagraph{Subparagraph}

% \ul = underline
% \st = strikethrough
% \hl = highlight
% \textbf = bold face
% \textit = italic face
% \textsl = slanted
% \textsf = sans serif

%%%%%%%%%%%%%%%%%%%%%%%%%%%%%%%%%%%%%%%%%%%%%%%%%%%%%%%%%%%%%%%%%%%%%%%%%

This chapter provides an introduction to this thesis.
In \cref{s:introduction-motivation} the general motivation is discussed.
The section is followed by outlining the research goals in \cref{s:introduction-researchgoals}.
\Cref{s:introduction-researchmethodology} gives an insight into research methods used in order to answer the raised research questions.
Finally, \cref{s:introduction-structureofthisthesis} gives an outlook of the structure of the remaining thesis.

\section{Motivation}
\label{s:introduction-motivation}



\section{Research Goals}
\label{s:introduction-researchgoals}

The goal of this research to analyze the correlation between sentiment of tweets and share movement of automotive companies.
This goal will be met by achieving the following objectives:

\begin{itemize}
    \item Determine companies, keywords and stock symbols to analyze
    \item Gather tweets and stock prices
    \item Normalization of tweets
    \item Determine sentiment of tweets
    \item Comparing sentiment time series with share prices
\end{itemize}

\section{Research Methodology}
\label{s:introduction-researchmethodology}

\section{Structure of this Thesis}
\label{s:introduction-structureofthisthesis}
