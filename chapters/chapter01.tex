%!TEX root = ../main.tex

% \section{Section Title}
% \subsection{Subsection}
% \subsubsection{Subsubsection}
% \paragraph{Paragraph}
% \subparagraph{Subparagraph}

% \ul = underline
% \st = strikethrough
% \hl = highlight
% \textbf = bold face
% \textit = italic face
% \textsl = slanted
% \textsf = sans serif

%%%%%%%%%%%%%%%%%%%%%%%%%%%%%%%%%%%%%%%%%%%%%%%%%%%%%%%%%%%%%%%%%%%%%%%%%

This chapter provides an introduction to this thesis.
In \cref{s:introduction-motivation} the general motivation is discussed.
The section is followed by outlining the research goals in \cref{s:introduction-researchgoals}.
\Cref{s:introduction-researchmethodology} gives an insight into research methods used in order to answer the raised research questions.
Finally, \cref{s:introduction-structureofthisthesis} gives an outlook of the structure of the remaining thesis.

\section{Motivation}
\label{s:introduction-motivation}

Many studies have been published which are trying to predict the stock market movement.
As the \ac{EMH} states that financial market movements depend on news, current events and product releases and all these factors will have significant impact on a company's stock value
\cite{fama1965behavior}.
Due the fact that news and current events are unpredictable stock market prices are following a random walk pattern and cannot predicted with more than 50\% accuracy
\cite{Pagolu2016a}.

Many internet users are microblogging nowadays.
Millions of messages are published daily on popular websites which provides microblogging services, such as Twitter, Tumblr and Facebook.
These published messages describing the personal life, opinions or current issues.
The more users are post about products and services they use the more microblogging websites become a valuable source of peoples opinions and sentiments.
Therefore, this data can be used for marketing, social studies and as a measure of public opinion
\cite{Patodkar2016a, Pagolu2016a}.

As most Twitter messages have a maximum length of 140 characters and speaks public opinion on a topic precisely
\cite{Pagolu2016a}.

\section{Research Goals}
\label{s:introduction-researchgoals}

According to the factors presented in \cref{s:introduction-motivation} the central research question can be formulated:
\emph{Can the stock market movements be explained by the public opinion extracted from Twitter?}

The goal of this research to analyze the correlation between sentiment of tweets and share movement of automotive companies.
This goal will be met by achieving the following objectives:

\begin{itemize}
    \item \textbf{G1} - Determine companies, keywords and stock symbols to analyze
    \item \textbf{G2} - Gather tweets and their sentiments and stock prices
    \item \textbf{G3} - Comparing sentiment time series with share prices
\end{itemize}

From definitions of goals and having the central question in mind the following research questions are set up:

\begin{itemize}
    \item \textbf{G1-Q1} - Which companies should be analyzed?
    \item \textbf{G1-Q2} - Which keywords should be used to find corresponding tweets?
    \item \textbf{G1-Q3} - Which company uses which stock symbol in order to retrieve share prices?
    \item \textbf{G2-Q4} - Why Twitter and not anything else?
    \item \textbf{G2-Q5} - In which way tweets can be collected?
    \item \textbf{G2-Q6} - In which way sentiments can be determined?
    \item \textbf{G2-Q7} - Which sentiments are present for various companies?
	\item \textbf{G3-Q8} - Can the time series of sentiments explain the share prices?
	\item \textbf{G3-Q9} - Can the time series of share prices explain the sentiments?
\end{itemize}

\section{Research Methodology}
\label{s:introduction-researchmethodology}

The research follows a structure deducted from ``evaluation techniques for systems analysis and design modelling methods'' by \citet{Siau2011} in which the authors try to show up the benefits and the shortcomings of different methods.
In the following the three main categories and their mapping to this thesis are shown:

\begin{description}
	\item[The \emph{theoretical and conceptual inquiry}] {
		establishes the theoretical background of this thesis.
		Through literature research definitions and types of stock market prediction, option mining and social networks are found.
	}
	
	\item[The \emph{case study}] {
		is needed to capture tweets on the internet.
		This is done by using a DMI TCAT installation
		\cite{Borra2014}.
	}
	
	\item[The \emph{metrics analysis}] {
		is used to compare the results of the case study with share prices of the automotive companies.
	}
\end{description}

As this thesis covers sentiments of people in a global context which is then compared to share prices in an economic context it can be classified as social science \citep{Recker2013}.
In the following the research actions, which have been undertaken to answer the research questions and fulfill the goals, are explained.

\begin{itemize}
	\item To find answers to the questions \textbf{Q1} to \textbf{Q5} literature research has been conducted.
	A keyword search has been performed on the literature search-engine \emph{Google Scholar} as well as library search.
	The retrieved literature is reviewed and based on the references new literature is obtained.
	
	\item With the theoretical background which has been obtained in answering the questions \textbf{Q1} to \textbf{Q6} a tweet collection system has been set up in order to answer the question \textbf{Q7}.
	This is done by setting up a open source tweet capturing system DMI TCAT system and evaluating the sentiment of the captured tweets.
	
	\item Questions \textbf{Q8} and \textbf{Q9} are answered through both literature research, which has been collected for the questions \textbf{Q1} to \textbf{Q6} and evaluated sentiments of the collected tweets for question \textbf{Q7}.
\end{itemize}

\section{Structure of this Thesis}
\label{s:introduction-structureofthisthesis}

This section is followed by the background \cref{c:background}, where the necessary theoretical background will be explained. 
In \cref{c:casestudy}, the setup of tweet collection is explained and the execution documented.
Afterwards, in \cref{c:analysis}, sentiments of collected tweets are determined and converted into a time series which is then compared to the time series of share prices.
Finally, in \cref{c:conclusion} the results of this work are summed up, and limitations and further points of interest are pointed out.
