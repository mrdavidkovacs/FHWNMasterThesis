This final section provides a short summary of literature research, analysis and comparison in \cref{s:conclusions-summary}.
\Cref{s:conclusions-discussion} discusses the results critically and shows up limitations and shortcomings. 
Finally, \cref{s:conclusions-future} provides possible future research topics.

\section{Summary of Results}
\label{s:conclusions-summary}

In this section we recall all defined research goals in \cref{s:introduction-researchgoals} on page \pageref{s:introduction-researchgoals} and give an answer to these.

% research goals

\subsection{G1 - Determine Companies, Keywords and Stock Symbols to Analyze}
\label{ss:conclusion-summary-g1}

% \item \textbf{G1-Q1} - Which companies should be analyzed?
% \item \textbf{G1-Q2} - Which keywords should be used to find corresponding tweets?
% \item \textbf{G1-Q3} - Which company uses which stock symbol in order to retrieve share prices?

\subsection{G2 - Gather Tweets and their Sentiments and Stock Prices}
\label{ss:conclusion-summary-g2}

% \item \textbf{G2-Q4} - Why Twitter and not anything else?
% \item \textbf{G2-Q5} - In which way tweets can be collected?
% \item \textbf{G2-Q6} - In which way sentiments can be determined?
% \item \textbf{G2-Q7} - Which sentiments are present for various companies?

\subsection{G3 - Comparing Sentiment Time Series with Share Prices}
\label{ss:conclusion-summary-g3}

% \item \textbf{G3-Q8} - Can the time series of sentiments explain the share prices?


\section{Discussion and Limitations}
\label{s:conclusions-discussion}

\section{Future Research}
\label{s:conclusions-future}

% Future work:
% Issues with DMI TCAT (no gaps!)
% sarcasm in tweets
% retweets?
% classification in positive, neutral and negative?