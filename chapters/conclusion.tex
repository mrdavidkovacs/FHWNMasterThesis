In a non-fiction book, a Conclusion is an ending section which states the concluding ideas and concepts of the preceding writing. This generally follows the body or perhaps a Afterword, and the conclusion may be followed by an Epilogue, Outro, Postscript, Appendix/Addendum, Glossary, Bibliography, Index, Errata, or a Colophon.

A common English usage misconception is that a paragraph has three to five sentences; single-word paragraphs can be seen in some professional writing, and journalists often use single-sentence paragraphs.

The crafting of clear, coherent paragraphs is the subject of considerable stylistic debate. Forms generally vary among types of writing. For example, newspapers, scientific journals, and fictional essays have somewhat different conventions for the placement of paragraph breaks.

English students are sometimes taught that a paragraph should have a topic sentence or "main idea", preferably first, and multiple "supporting" or "detail" sentences which explain or supply evidence. One technique of this type, intended for essay writing, is known as the Schaffer paragraph. For example, the following excerpt from Dr. Samuel Johnson's Lives of the English Poets, the first sentence is the main idea: that Joseph Addison is a skilled "describer of life and manners". The succeeding sentences are details that support and explain the main idea in a specific way.


% Future work:
% Issues with DMI TCAT (no interpolation!)
% sarcasm in tweets
% retweets?
% classification in positive, neutral and negative?