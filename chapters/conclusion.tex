This final section provides a short summary of literature research, analysis and comparison in \cref{s:conclusions-summary}.
\Cref{s:conclusions-discussion} discusses the results critically and shows up limitations and shortcomings. 
Finally, \cref{s:conclusions-future} provides possible future research topics.

\section{Summary of Results}
\label{s:conclusions-summary}

In this section we recall all defined research goals in \cref{s:introduction-researchgoals} on page \pageref{s:introduction-researchgoals} and give an answer to these.
In the following each research goal is stated and a short summary of the results is given in each subsection.

% research goals

\subsection{G1 - Determine Companies, Keywords and Stock Symbols to Analyze}
\label{ss:conclusion-summary-g1}

% \item \textbf{G1-Q1} - Which companies should be analyzed?
% \item \textbf{G1-Q2} - Which keywords should be used to find corresponding tweets?
% \item \textbf{G1-Q3} - Which company uses which stock symbol in order to retrieve share prices?

The research goal 1 has been answered in \cref{s:casestudy-companieskeywords} (see page \pageref{s:casestudy-companieskeywords}).
In the following a short summary is given.
First the question of which companies should be analyzed must be answered.
In literature there were many references to automotive companies and therefore the decision has been made to analyze these companies.
To find the corresponding companies the five biggest car manufacturing companies have been selected.
This selection has been made based on the survey \emph{World Motor Vehicle Production 2016} \citep{OICA2016}. 
The resulting companies are depicted in \cref{tab:casestudy-brands}.

Secondly, a list of keywords must be composed in order to search for tweets for the specific companies.
As these companies own several customer facing car manufacturing brands these will be used as keywords too.
Therefore, the \cref{tab:casestudy-brands} also contains all customer facing car manufacturing brands for the top five companies.

Thirdly, the companies must be traded on any stock exchange market and we have to research the stock symbols and the market in order to retrieve their stock prices.
The results of this research are depicted in \cref{tab:casestudy-companies-counts-and-symbols}.

\subsection{G2 - Gather Tweets and their Sentiments and Stock Prices}
\label{ss:conclusion-summary-g2}

% \item \textbf{G2-Q4} - Why Twitter and not anything else? (2.3)
% \item \textbf{G2-Q5} - In which way tweets can be collected?
% \item \textbf{G2-Q6} - In which way sentiments can be determined?
% \item \textbf{G2-Q7} - Which sentiments are present for various companies?

The research goal 2 consists of four research questions.
First the decision for using Twitter as social media platform of our choice has been made as the characters are limited for each post and therefore the probability of a single topic per tweet is higher than on other platforms.
Furthermore, several other papers in literature have been using Twitter as reliable source.
This has been discussed in \cref{s:background-socialnetworks}.

Secondly, there is the questions how tweets can be collected from twitter which has been answered in \cref{ss:casestudy-gatherdata-tweets}.
Three different possibilities of tweet collection could be identified:

\begin{description}
    \item[Official Twitter Search \ac{API}]
        follows the pull principle. 
        The user requests something and the \ac{API} sends a response.
        But there were some serious limitations which caused that this way cannot be used.

    \item[Twitter search on website]
        does not have these limitations as the search \ac{API}.
        But it is no easy task to download and save search results from the official Twitter page.
        Therefore, also this possibility has been omitted due the difficulties.    
    
    \item[DMI TCAT] 
        is a toolset for capturing and analyzing tweets.
        It follows a different approach than the search \ac{API}.
        It supports the official streaming \ac{API} which follows a push principle and enables its users to capture tweets using up to 400 keywords.
        Therefore, the decision has been made to use DMI TCAT to capture tweets.

\end{description}

Thirdly, research has been made in which way sentiments can be determined (see \cref{s:casestudy-normalization} and \cref{s:casestudy-sentiment} for more details).
Most sentiment detection algorithms are performing their work better if the text source has been normalized in some way.
Normalization includes lower casing of text, applying stopwords, lemmatization, stemming and minor enhancements.
All of these normalization techniques have been applied to the collection of tweets which was a quite time consuming task as there were over 16 million tweets captured (see \cref{tab:casestudy-companies-numberoftweets}).

Four sentiment detection algorithms have been selected for the case study: \tb{}, \nb{}, \me{} and \svm{}.
In order to standardize the training and classification of each algorithm the \emph{scikit-learn} framework has been used.
The framework includes all algorithms and a so called \emph{pipeline} to perform this standardization for training and classification purposes.
The full script to train and analyze tweet sentiments can be found in the appendix (see \cref{lst:appendix-tweeets_analyzer}).

Fourthly, an analysis has been performed which sentiments are present in which company.
This is done in \cref{s:analysis-sentiments} beginning on page \pageref{s:analysis-sentiments}.
An overview of the tweet sentiments are depicted in \cref{tab:analysis-sentiments-general}.
It is shown that almost \SI{75}{\percent} of all tweets were rated as positive across all classifiers but there are differences in percentages by classifiers and company.
The \tb{} classifier for example is very optimistic whereas the \nb{} classifier is relatively pessimistic.
The same holds true for certain companies: \SI{80}{\percent} of all \gm{} tweets were positive whereas only $\frac{2}{3}$ of \vw{} tweets were positive.

\subsection{G3 - Comparing Sentiment Time Series with Share Prices}
\label{ss:conclusion-summary-g3}

% \item \textbf{G3-Q8} - Can the time series of sentiments explain the share prices?


\section{Discussion and Limitations}
\label{s:conclusions-discussion}

\section{Future Research}
\label{s:conclusions-future}

% Future work:
% Issues with DMI TCAT (no gaps!)
% sarcasm in tweets
% retweets?
% classification in positive, neutral and negative?