This final section provides a short summary of literature research, analysis and comparison in \cref{s:conclusions-summary}.
\Cref{s:conclusions-discussion} discusses the results critically and shows up limitations and shortcomings. 
Finally, \cref{s:conclusions-future} provides possible future research topics.

\section{Summary of Results}
\label{s:conclusions-summary}

In this section we recall all defined research goals in \cref{s:introduction-researchgoals} on page \pageref{s:introduction-researchgoals} and give an answer to these.
In the following each research goal is stated and a short summary of the results is given in each subsection.

% research goals

\subsection{G1 - Determine Companies, Keywords and Stock Symbols to Analyze}
\label{ss:conclusion-summary-g1}

% \item \textbf{G1-Q1} - Which companies should be analyzed?
% \item \textbf{G1-Q2} - Which keywords should be used to find corresponding tweets?
% \item \textbf{G1-Q3} - Which company uses which stock symbol in order to retrieve share prices?
The research goal 1 has been answered in \cref{s:casestudy-companieskeywords} (see page \pageref{s:casestudy-companieskeywords}).
In the following a short summary is given.
First the question of which companies should be analyzed must be answered.
In literature there were many references to automotive companies and therefore the decision has been made to analyze these companies.
To find the corresponding companies the five biggest car manufacturing companies have been selected.
This selection has been made based on the survey \emph{World Motor Vehicle Production 2016} \citep{OICA2016}. 
The resulting companies are depicted in \cref{tab:casestudy-brands}.

Secondly, a list of keywords must be composed in order to search for tweets for the specific companies.
As these companies own several customer facing car manufacturing brands these will be used as keywords too.
Therefore, the \cref{tab:casestudy-brands} also contains all customer facing car manufacturing brands for the top five companies.

Thirdly, the companies must be traded on any stock exchange market and we have to research the stock symbols and the market in order to retrieve their stock prices.
The results of this research are depicted in \cref{tab:casestudy-companies-counts-and-symbols}.

\subsection{G2 - Gather Tweets and their Sentiments and Stock Prices}
\label{ss:conclusion-summary-g2}

% \item \textbf{G2-Q4} - Why Twitter and not anything else? (2.3)
% \item \textbf{G2-Q5} - In which way tweets can be collected?
% \item \textbf{G2-Q6} - In which way sentiments can be determined?
% \item \textbf{G2-Q7} - Which sentiments are present for various companies?

\subsection{G3 - Comparing Sentiment Time Series with Share Prices}
\label{ss:conclusion-summary-g3}

% \item \textbf{G3-Q8} - Can the time series of sentiments explain the share prices?


\section{Discussion and Limitations}
\label{s:conclusions-discussion}

\section{Future Research}
\label{s:conclusions-future}

% Future work:
% Issues with DMI TCAT (no gaps!)
% sarcasm in tweets
% retweets?
% classification in positive, neutral and negative?