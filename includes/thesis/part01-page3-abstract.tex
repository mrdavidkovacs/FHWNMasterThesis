%!TEX root = ../thesis.tex

\newcommand*{\AbstractHead}[1]{%
{\noindent\color{header-blue}\Large\textbf{#1}}
\vspace{10pt}\\
}% These command is used to create the Date and Signature fields.

\newcommand*{\SomeSpace}{%
\vspace{\baselineskip}
}

\AbstractHead{Abstract in English}
\noindent
\normalsize
Despite the theory of efficient market hypothesis some researchers suggest that stock prices can at least partly predicted.
Nowadays Twitter contains a public opinion for every topic as the social media platform gets more popular.

This thesis extracts tweets using DMI-TCAT for five automotive manufacturing companies.
In total \num{11565283} tweets have been captured in the time frame between \printdate{2018-02-28} and \printdate{2018-09-07}.
Afterwards several sentiment detection algorithms (\tb{}, \nb{}, \me{} and \svm{}) have been applied to extract indicators for the public opinion about the five companies.
The indicators were compared with the share price of the companies using Granger analysis.

It is shown that the share prices of \ford{} and \hyundai{} cannot be predicted very well as only one test of the Granger analysis was significant.
Furthermore, both \svm{} and \nb{} classifiers outperformed the \tb{} classifier, which was the reference value during the analysis.

% Many studies have been published which are trying to predict the stock market movement.
% The efficient market hypothesis states that the movements depend on news which are per definition not predictable the stock will follow a random walk pattern.
% Nevertheless some researchers suggest that stock prices are at least partly predictable. 
% On the other hand there are many internet users microblogging today on services like Twitter.
% As the count of characters is limited on Twitter users have to be concise about the topic to write.

\SomeSpace
\AbstractHead{Keywords}
\normalsize
\noindent
Efficient Market Hypothesis, Stock Price, Twitter, Sentiment Classification

\glsresetall
\SomeSpace

\AbstractHead{Abstract in German}
\noindent
\normalsize
Trotz der Theorie der Effizienten Markt Hypothese meinen manche Forscher, dass Aktienkurse zumindest teilweise vorhergesagt werden können.
Heutzutage ist auf Plattformen wie Twitter zu jedem beliebigen Thema eine Meinung zu finden.

Diese vorliegende Arbeit extrahiert Tweets mittels DMI-TCAT für fünf Automobilhersteller.
Insgesamt wurden 11565283 Tweets in einem Zeitraum zwischen 28.02.2018 und 07.09.2018 erfasst.
Im Anschluss wurden vier verschiedene Sentiment Klassifizierungs Alogrithmen (\tb{}, \nb{}, \me{} and \svm{}) auf die gesammelten Tweets angewandt, um Indikatoren für die öffentliche Stimmung zu den jeweiligen Firmen zu extrahieren.
Diese Indikatoren wurden mit den Aktienkursen des jeweiligen Unternehmens mittels Granger Analyse verglichen.

Dabei hat sich gezeigt, dass die Aktienkurse von \ford{} und \hyundai{} sehr schlecht vorhergesagt werden konnten, da nur ein einziger Granger Test signifikant war.
Weiters wurde gezeigt, dass die \svm{} und \nb{} Klassifizierungsalgorithmen dem Referenzalgorithmus \tb{} überlegen sind.

\SomeSpace
\AbstractHead{Keywords}
\normalsize
\noindent
Effiziente Markt Hypothese, Aktienkurs, Twitter, Sentiment Klassifizierung

\glsresetall