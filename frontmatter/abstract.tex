
\newcommand*{\AbstractHead}[1]{%
{\noindent\color{header-blue}\Large\textbf{#1}}
\vspace{10pt}\\
}% These command is used to create the Date and Signature fields.

\newcommand*{\SomeSpace}{%
\vspace{\baselineskip}
}

\AbstractHead{Abstract in English}
\noindent
\normalsize
Despite the theory of efficient market hypothesis some researchers suggest that stock prices can at least partly predicted.
Nowadays Twitter contains a public opinion for every topic as it gets more popular.

This Thesis extracts tweets from Twitter using DMI-TCAT for five automotive manufacturing companies: \ford{}, \gm{}, \hyundai{}, \toyota{} and \vw{}.
In total \num{11565283} tweets have been captured in the time frame between \printdate{2018-02-28} and \printdate{2018-09-07}.
Afterwords several sentiment detection algorithms (\tb{}, \nb{}, \me{} and \svm{}) have been applied to extract indicators for the public opinion.
The indicators are compared with the share price of the companies using Granger analysis.

It is shown that the share prices of \ford{} and \hyundai{} cannot be predicted very well as only one test of the Granger analysis was significant.
Furthermore, both \svm{} and \nb{} classifiers outperformed the \tb{} classifier which was the reference value during the analysis.

% Many studies have been published which are trying to predict the stock market movement.
% The efficient market hypothesis states that the movements depend on news which are per definition not predictable the stock will follow a random walk pattern.
% Nevertheless some researchers suggest that stock prices are at least partly predictable. 
% On the other hand there are many internet users microblogging today on services like Twitter.
% As the count of characters is limited on Twitter users have to be concise about the topic to write.

\SomeSpace
\AbstractHead{Keywords (at least 3, max. 6)}
\normalsize
\noindent
Efficient Market Hypothesis, Twitter, Sentiment Classification

\SomeSpace

\AbstractHead{Abstract in German}
\noindent
\normalsize
%Trotz der Therorie der Effizienten Markt Hypothese 
Standard Standard Standard Standard Standard Standard Standard Standard Standard Standard Standard Standard Standard Standard Standard Standard Standard Standard Standard Standard Standard Standard Standard Standard Standard Standard Standard Standard Standard Standard Standard Standard Standard Standard Standard Standard Standard Standard Standard Standard Standard Standard Standard Standard Standard Standard Standard Standard Standard Standard Standard Standard Standard Standard Standard Standard Standard Standard Standard Standard Standard Standard Standard Standard Standard Standard Standard Standard Standard Standard Standard Standard Standard Standard Standard Standard Standard Standard Standard Standard Standard Standard Standard Standard Standard Standard Standard Standard Standard Standard Standard Standard Standard Standard Standard Standard Standard Standard Standard Standard Standard Standard Standard Standard Standard Standard Standard Standard Standard 

\SomeSpace
\AbstractHead{Keywords (at least 3, max. 6)}
\normalsize
\noindent
Effiziente Markt Hypothese, Twitter, Sentiment Klassifizierung