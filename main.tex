%This is a proposed template for a master thesis report for the FH Wiener Neustadt. The present document is free.

%You can redistribute it and/or modify it under the terms of the GNU General Public License as published by the Free Software Foundation; either version 2 of the License, or (at your option) any later version.

% Author: Joan Miralles
% Datum: 05/04/2015
% Latest update: 11/07/2016
%version: 2.2

%   Study Thesis:   Write here your title
%   Institution:    Fachhochschule Wiener Neustadt
%   Topic:          Whatever you want to write about
%   Author:         Name Name Name
%   Date:           01/04/2015
%   Version:        1.0    

        %----------------------------------------------------%
\documentclass[oneside,12pt,bibliography=totoc]{scrbook}

\subject{Master's Thesis}
\title{An analysis of share movement and sentiment of tweets: detailed analysis of automotive companies.}
%\subtitle{A fiction story telling the truth}
\author{David Kovacs, MSc}
%\publishers{Published by the publisher\\ at a secret location}
%\extratitle{Stories of ducks}
%\uppertitleback{\vspace{3cm}I want to thank Scrooge MacDuck for his financial Support}
\date{\today}

%'oneside' specifies that the pages will be printed on one sides of a every page. If you want to print on both sides of your pages use the parameter 'twoside'. Remember to order the printing job on one side of every page!!!


    %%%%------------------PREAMBLE----------------%%%%
    %%----------------------------------------------%%

    %%%%%-------------------PACKAGES-----------------%%%%% 
    
%\usepackage[utf8]{inputenc}
\usepackage[a4paper,top=25mm,bottom=30mm,left=30mm,right=25mm]{geometry}   %allows to change margins and distances within the page's frame. Be careful with this tool. Remember to delete the frame!!!!

%\usepackage{showframe}
%\usepackage[english]{babel}    %the package babel is not necessary as long as you write your thesis in English.
%\usepackage{csquotes}
\usepackage{graphicx}   %loads the graphics package, necessary for pictures
\usepackage{url}
%\usepackage{wallpaper}
\usepackage[colorlinks]{hyperref}   %Automatically turn all your internal references into hyperlinks.
\makeatletter
\hypersetup{pdfinfo={
  Title=\@title,
  Author=\@author,
  Subject=Master's Thesis,
  Keywords={EMH,Twitter,Sentiment}}
}
\makeatother

\usepackage[toc,page]{appendix}
\usepackage[final]{pdfpages}
\usepackage{soul}       %provides hyphenatable letterspacing, underlining and some derivatives such as everstriking and highlighting.
%\usepackage[intoc]{nomencl}   %helps to format a nomenclature.
%\makenomenclature

\usepackage[nomain,acronym,toc,shortcuts]{glossaries}

\usepackage[UKenglish]{isodate} %Provides the format for the date on the title page. Read the package documentation for all the possible date formats.

\usepackage{array}
\usepackage{longtable}

\usepackage{tikz}

%\usepackage{lipsum}
\usepackage{caption}
%\usepackage{abstract}   %This package allows to modify the Abstract style, position and font. Defines the abstract automatically.
\usepackage{supertabular}
\usepackage{tabto}

\usepackage{fontspec}
\setmainfont{MyFont.TTF}%
[Ligatures=TeX,
    Path=fonts/,
    BoldFont=MyFont_Bold.TTF,
    ItalicFont=MyFont_Italic.TTF,
    BoldItalicFont=MyFont_BoldItalic.TTF]

\usepackage{xcolor}
\definecolor{header-blue}{RGB}{22,48,114}

\setcounter{secnumdepth}{4}

\usepackage{titlesec}

\titleformat{\chapter}[hang]{\bfseries\Large\color{header-blue}}{\makebox[.63cm][l]{\thechapter.}}{0pt}{}[]
\titlespacing*{\chapter}{0pt}{18pt}{12pt}

\titleformat{\section}[hang]{\bfseries\large\color{header-blue}}{\makebox[1.27cm][l]{\thesection}}{0pt}{}[]
\titlespacing*{\section}{0pt}{6pt}{12pt}

\titleformat{\subsection}[hang]{\bfseries\normalsize\color{header-blue}}{\makebox[1.5cm][l]{\thesubsection}}{0pt}{}[] 
\titlespacing*{\subsection}{0pt}{6pt}{12pt}

\titleformat{\subsubsection}[hang]{\bfseries\normalsize\color{header-blue}}{\makebox[1.5cm][l]{\thesubsubsection}}{0pt}{}[]
\titlespacing*{\subsubsection}{0pt}{6pt}{12pt}

\titleformat{\paragraph}[hang]{\bfseries\normalsize\color{header-blue}}{\makebox[2.5cm][l]{\theparagraph}}{0pt}{}[]
\titlespacing*{\paragraph}{0pt}{6pt}{12pt}

%\usepackage{footnote}      %Just in case you want to include footnotes
%\usepackage{indentfirst}    %Indent in every first paragraph after \chapter and \section.
\usepackage[onehalfspacing]{setspace}      %Change spacing easily with simple commands

\usepackage{amsmath}       %This package introduces several new commands that are more powerful and flexible than the ones provided by LaTeX. Uncomment them if you are using complex mathematical notation.
\usepackage{mathtools}     %The mathtools package fixes some amsmath quirks and adds some useful settings, symbols, and environments to amsmath.

\usepackage{cleveref}
\usepackage{epigraph}
\usepackage{pgffor}

\graphicspath{{images/}}    %Direction where LaTeX will look for the images.
\DeclareGraphicsExtensions{.pdf,.png,.jpg,.mps}

\usepackage[backend=bibtex,style=numeric,citestyle=numeric-comp]{biblatex}
\addbibresource{references.bib}     %Imports bibliography file

\clubpenalty = 10000 
\widowpenalty = 10000
\displaywidowpenalty = 10000

\definecolor{mygreen}{rgb}{0,0.6,0}
\definecolor{mygray}{rgb}{0.5,0.5,0.5}
\definecolor{mylightgray}{rgb}{0.9,0.9,0.9}
\definecolor{mymauve}{rgb}{0.58,0,0.82}

\usepackage{listings}

\lstset{ %
  %backgroundcolor=\color{mylightgray},   % choose the background color; you must add \usepackage{color} or \usepackage{xcolor}
  basicstyle=\linespread{0.8}\footnotesize,        % the size of the fonts that are used for the code
	%basicstyle=\small\sffamily,
  breakatwhitespace=false,         % sets if automatic breaks should only happen at whitespace
  breaklines=true,                 % sets automatic line breaking
  captionpos=t,                    % sets the caption-position to bottom
  commentstyle=\color{mygreen},    % comment style
  deletekeywords={...},            % if you want to delete keywords from the given language
  escapeinside={\%*}{*)},          % if you want to add LaTeX within your code
  extendedchars=true,              % lets you use non-ASCII characters; for 8-bits encodings only, does not work with UTF-8
  frame=tb,                    % adds a frame around the code
  keepspaces=true,                 % keeps spaces in text, useful for keeping indentation of code (possibly needs columns=flexible)
  keywordstyle=\color{blue},       % keyword style
  %language=Octave,                 % the language of the code
  morekeywords={*,...},            % if you want to add more keywords to the set
  numbers=left,                    % where to put the line-numbers; possible values are (none, left, right)
  numbersep=5pt,                   % how far the line-numbers are from the code
  numberstyle=\tiny\color{mygray}, % the style that is used for the line-numbers
  rulecolor=\color{black},         % if not set, the frame-color may be changed on line-breaks within not-black text (e.g. comments (green here))
  showspaces=false,                % show spaces everywhere adding particular underscores; it overrides 'showstringspaces'
  showstringspaces=false,          % underline spaces within strings only
  showtabs=false,                  % show tabs within strings adding particular underscores
  stepnumber=1,                    % the step between two line-numbers. If it's 1, each line will be numbered
  stringstyle=\color{mymauve},     % string literal style
  tabsize=2,                       % sets default tabsize to 2 spaces
  title=\lstname                   % show the filename of files included with \lstinputlisting; also try caption instead of title
}

\renewcommand{\labelitemii}{-}
\renewcommand{\labelitemi}{--}

\newcolumntype{!}{>{\global\let\currentrowstyle\relax}}
\newcolumntype{^}{>{\currentrowstyle}}
\newcommand{\rowstyle}[1]{\gdef\currentrowstyle{#1}%
  #1\ignorespaces
}

    %%%------------- Page Layout Settings ----------------------%
%%---Just in case you want to customize the page layout-----
%\setlength{\topmargin}{0mm} 
%\setlength{\textheight}{225mm} 
%\setlength{\oddsidemargin}{4,6mm} 
%\setlength{\evensidemargin}{4,6mm}
%\setlength{\textwidth}{150mm} 
%\setlength{\headheight}{15pt}

    %%%%----------------Paragraphs formatting---------------------%%%%%
%\linespread{1.0}    %Line spacing, only accepts values 1.0, 1.3 and 1.6. If set here, it will apply to the whole document, including ToC, Lists...
%\setlength{\baselineskip}{5mm} %%minimum space between the bottom of two successive lines in a paragraph
%\setlength{\parindent}{1.25cm}  %Determining paragraph indentation
%\setlength{\parskip}{2mm}   %Determining space between paragraph and preceeding text. This can be set in the preamble or later in the document part, but be careful. If set here, it will apply to the whole document, including ToC, Lists...

    %------------------------- Caption Setup -----------------------%
%\captionsetup{margin=10pt, font=small, labelfont=bf, labelsep=period}

    %-------------------Footnote Setup---------------------%
%\makesavenoteenv{tabular}

    %%%------------------END OF THE PREAMBLE-----------------%%%
    %%--------------------------------------------------------%%

    %------------------------DOCUMENT-----------------------------%

\makeglossaries
%!TEX root = ../main.tex

\newacronym{EMH}{EMH}{Efficient Market Hypothesis}
\newacronym{ME}{ME}{Maximum Entropy}
\newacronym{NB}{NB}{Naive Bayes}
\newacronym{NLP}{NLP}{Natural Language Processing}
\newacronym{SVM}{SVM}{Support Vector Machines}
\newacronym{TP}{TP}{True Positive}
\newacronym{TN}{TN}{True Negative}
\newacronym{FP}{FP}{False Positive}
\newacronym{FN}{FN}{False Negative}
\newacronym{bow}{bow}{bag of words}

\begin{document}

    %----------------------------FRONT MATTER---------------------%
\frontmatter

%The next two commands are defined by the user in order to create a table insid the front page where to place the details of the thesis (name, matriculation number, master program and supervisor)

\newcommand\textline[4][t]{%
  \par\smallskip\noindent\parbox[#1]{.333\textwidth}{\large\raggedright\texttt{}#2}%
  \parbox[#1]{.333\textwidth}{\large\centering\textbf{#3}}%
  \parbox[#1]{.333\textwidth}{\raggedleft\texttt{#4}}\par\smallskip%
}

\newcommand\specialtxtl[4][t]{%
  \par\smallskip\noindent\parbox[#1]{.2\textwidth}{\large\raggedright\texttt{}#2}%
  \parbox[#1]{.6\textwidth}{\LARGE\centering\textbf{#3}}%
  \parbox[#1]{.2\textwidth}{\raggedleft\texttt{#4}}\par\smallskip%
}

%%%%!!!! IMPORTANT, if your name is too long and the program splits it up in two different lines, you can play with the 'textwidth' paramenter in the \newgeometry{} command as well as the text width on the command \textline, above this lines. I recommend you to do so only if necessary, otherwise you might screw up the layout of the page.

\begin{titlepage}
    \newgeometry{voffset=-15.2mm,textwidth=430pt,headheight=0pt,textheight=740pt} % BE VERY CAREFUL with the geometry definitions here. This might screw up the layout of the page. The only ones which is not set to default is `voffset` and `textwidth`
  
    \centering

    \makebox[0pt]{\includegraphics{title_header}}   %if the quality is not good enough when converting the PDF, you might want to try a vector image instead.
    
      \begin{framed}
        \LARGE
        \vspace{1cm}
        \textbf{Experimental Characterization of some new technology we have to develop, with applications in space walks}
        \vspace{1cm}
      \end{framed}

      \LARGE
      \vspace{.7cm}
      \textbf{Master Thesis}

      \vspace{1.0cm}
      \textline[t]{Submitted by:}{\LARGE Author Name}{}

      \textline[t]{Matriculation Number:}{\LARGE 12345678901}{}

      \vspace{1.0cm}

      \specialtxtl[t]{at:}{Master's program}{}

      \textbf{``Aerospace Engineering''}

      \vspace{2.5cm}
      \specialtxtl[t]{Supervisor:}{Supervisor 1 Name}{}
      \textbf{\LARGE Supervisor 2 Name}

      \vspace{\fill}
      \normalsize
      \raggedright Wiener Neustadt, \today
      
      \rule{\textwidth}{.5pt}
      
\end{titlepage}

\newcommand*{\SignatureAndDate}[1]{%
    \par\makebox[4cm]{Wiener Neustadt,} \makebox[3cm]{\hrulefill}  \hfill\makebox[6cm]{\hrulefill}%
    \par\makebox[4cm]{} \makebox[3cm]{\centering Date}     \hfill\makebox[3.5cm][t]{Signature}%
}% These command is used to create the Date and Signature fields.

\chapter*{Declaration of Integrity}

\thispagestyle{empty}

\noindent\rule{\textwidth}{.5pt}
I hereby confirm,
\begin{enumerate}
    \item That I have written the Master’s thesis at hand independently, that I have not used any sources or materials other than those stated, nor availed myself of any unauthorized resources, and
    \item That I have not submitted this Master’s thesis in any form as an examination paper before, neither in this country, nor abroad, and
    \item That the electronic copy of this Master’s thesis and the printed versions are identical.
\end{enumerate}

    \vspace{1.0cm}

\SignatureAndDate{}


\newcommand*{\AbstractHead}[1]{%
{\noindent\color{header-blue}\Large\textbf{#1}}
\vspace{10pt}\\
}% These command is used to create the Date and Signature fields.

\newcommand*{\SomeSpace}{%
\vspace{\baselineskip}
}

\AbstractHead{Abstract in English}
\noindent
\normalsize
Standard Standard Standard Standard Standard Standard Standard Standard Standard Standard Standard Standard Standard Standard Standard Standard Standard Standard Standard Standard Standard Standard Standard Standard Standard Standard Standard Standard Standard Standard Standard Standard Standard Standard Standard Standard Standard Standard Standard Standard Standard Standard Standard Standard Standard Standard Standard Standard Standard Standard Standard Standard Standard Standard Standard Standard Standard Standard Standard Standard Standard Standard Standard Standard Standard Standard Standard Standard Standard Standard Standard Standard Standard Standard Standard Standard Standard Standard Standard Standard Standard Standard Standard Standard Standard Standard Standard Standard Standard Standard Standard Standard Standard Standard Standard Standard Standard Standard Standard Standard Standard Standard Standard Standard Standard Standard Standard Standard Standard 

\SomeSpace
\AbstractHead{Keywords (at least 3, max. 6)}
\normalsize
\noindent
Standard Standard Standard Standard Standard Standard

\SomeSpace

\AbstractHead{Abstract in German}
\noindent
\normalsize
Standard Standard Standard Standard Standard Standard Standard Standard Standard Standard Standard Standard Standard Standard Standard Standard Standard Standard Standard Standard Standard Standard Standard Standard Standard Standard Standard Standard Standard Standard Standard Standard Standard Standard Standard Standard Standard Standard Standard Standard Standard Standard Standard Standard Standard Standard Standard Standard Standard Standard Standard Standard Standard Standard Standard Standard Standard Standard Standard Standard Standard Standard Standard Standard Standard Standard Standard Standard Standard Standard Standard Standard Standard Standard Standard Standard Standard Standard Standard Standard Standard Standard Standard Standard Standard Standard Standard Standard Standard Standard Standard Standard Standard Standard Standard Standard Standard Standard Standard Standard Standard Standard Standard Standard Standard Standard Standard Standard Standard 

\SomeSpace
\AbstractHead{Keywords (at least 3, max. 6)}
\normalsize
\noindent
Standard Standard Standard Standard Standard Standard

\normalsize

%%--------------It's your choice to include a dedication and/or acknowledgement---------------%%

%\chapter*{Dedication}
%\input{frontmatter/dedication}

%\chapter*{Acknowledgement}
%\input{frontmatter/acknowledgement}

        %%------------------------TABLE OF CONTENTS AND LISTS-------------------%%
        %%----------------------------------------------------------------------%%
        
\tableofcontents    %\thispagestyle{fancy}

%\chapter*{List of Acronyms}
%\addcontentsline{toc}{chapter}{List of Acronyms}
%\printglossary[type=\acronymtype]
\printglossary[type=\acronymtype,title=List of Acronyms,style=long]

\listoffigures  %\thispagestyle{fancy}
\addcontentsline{toc}{chapter}{List of Figures}

\listoftables   %\thispagestyle{fancy}
\addcontentsline{toc}{chapter}{List of Tables}

\mainmatter

        %--------------------BODY OF THE THESIS--------------------------%
        %%--------------------------------------------------------------%%
%\cleardoublepage
\chapter{Introduction}    
    %!TEX root = ../main.tex

% \section{Section Title}
% \subsection{Subsection}
% \subsubsection{Subsubsection}
% \paragraph{Paragraph}
% \subparagraph{Subparagraph}

% \ul = underline
% \st = strikethrough
% \hl = highlight
% \textbf = bold face
% \textit = italic face
% \textsl = slanted
% \textsf = sans serif

%%%%%%%%%%%%%%%%%%%%%%%%%%%%%%%%%%%%%%%%%%%%%%%%%%%%%%%%%%%%%%%%%%%%%%%%%

This chapter provides an introduction to this thesis.
In \cref{s:introduction-motivation} the general motivation is discussed.
The section is followed by outlining the research goals in \cref{s:introduction-researchgoals}.
\Cref{s:introduction-researchmethodology} gives an insight into research methods used in order to answer the raised research questions.
Finally, \cref{s:introduction-structureofthisthesis} gives an outlook of the structure of the remaining thesis.

\section{Motivation}
\label{s:introduction-motivation}

Many studies have been published which are trying to predict the stock market movement \citep[see][]{Bollen2011a,Mittal2012a,Nguyen2015a,Pagolu2016a,Zhang2011a}.
As the \ac{EMH} states that financial market movements depend on news, current events and product releases and all these factors will have significant impact on a company's stock value
\citep{fama1965behavior}.
Due the fact that news and current events are unpredictable stock market prices are following a random walk pattern and cannot predicted with more than \SI{50}{\percent} accuracy
\citep{Pagolu2016a}.

Many internet users are microblogging nowadays.
Millions of messages are published daily on popular websites which provides microblogging services, such as Twitter, Tumblr and Facebook.
These published messages describing the personal life, opinions or current issues.
The more users are post about products and services they use the more microblogging websites become a valuable source of peoples opinions and sentiments.
Therefore, this data can be used for marketing, social studies and as a measure of public opinion
\citep{Patodkar2016a, Pagolu2016a}.

As most Twitter messages have a maximum length of 140 characters and speaks public opinion on a topic precisely
\citep{Pagolu2016a}.

\section{Research Goals}
\label{s:introduction-researchgoals}

According to the factors presented in \cref{s:introduction-motivation} the central research question can be formulated:
\emph{Can the stock market movements be explained by the public opinion extracted from Twitter?}

The goal of this research to analyze the correlation between sentiment of tweets and share movement of automotive companies.
This goal will be met by achieving the following objectives:

\begin{itemize}
    \item \textbf{G1} - Determine companies, keywords and stock symbols to analyze
    \item \textbf{G2} - Gather tweets and their sentiments and stock prices
    \item \textbf{G3} - Comparing sentiment time series with share prices
\end{itemize}

From definitions of goals and having the central question in mind the following research questions are set up:

\begin{itemize}
    \item \textbf{G1-Q1} - Which companies should be analyzed?
    \item \textbf{G1-Q2} - Which keywords should be used to find corresponding tweets?
    \item \textbf{G1-Q3} - Which company uses which stock symbol in order to retrieve share prices?
    \item \textbf{G2-Q4} - Why Twitter and not anything else?
    \item \textbf{G2-Q5} - In which way tweets can be collected?
    \item \textbf{G2-Q6} - In which way sentiments can be determined?
    \item \textbf{G2-Q7} - Which sentiments are present for various companies?
	\item \textbf{G3-Q8} - Can the time series of sentiments explain the share prices?
	%\item \textbf{G3-Q9} - Can the time series of share prices explain the sentiments?
\end{itemize}

\section{Research Methodology}
\label{s:introduction-researchmethodology}

The research follows a structure deducted from ``evaluation techniques for systems analysis and design modelling methods'' by \citet{Siau2011} in which the authors try to show up the benefits and the shortcomings of different methods.
In the following the three main categories and their mapping to this thesis are shown:

\begin{description}
	\item[The \emph{theoretical and conceptual inquiry}] {
		establishes the theoretical background of this thesis.
		Through literature research definitions and types of stock market prediction, option mining and social networks are found.
	}
	
	\item[The \emph{case study}] {
		is needed to capture tweets on the internet.
		This is done by using a DMI TCAT installation
		\citep{Borra2014}.
	}
	
	\item[The \emph{metrics analysis}] {
		is used to compare the results of the case study with share prices of the automotive companies.
	}
\end{description}

As this thesis covers sentiments of people in a global context which is then compared to share prices in an economic context it can be classified as social science \citep{Recker2013}.
In the following the research actions, which have been undertaken to answer the research questions and fulfill the goals, are explained.

\begin{itemize}
	\item To find answers to the questions \textbf{Q1} to \textbf{Q5} literature research has been conducted.
	A keyword search has been performed on the literature search-engine \emph{Google Scholar} as well as library search.
	The retrieved literature is reviewed and based on the references new literature is obtained.
	
	\item With the theoretical background which has been obtained in answering the questions \textbf{Q1} to \textbf{Q6} a tweet collection system has been set up in order to answer the question \textbf{Q7}.
	This is done by setting up a open source tweet capturing system DMI TCAT system and evaluating the sentiment of the captured tweets.
	
	\item Question \textbf{Q8} is answered through both literature research, which has been collected for the questions \textbf{Q1} to \textbf{Q6} and evaluated sentiments of the collected tweets for question \textbf{Q7}.
\end{itemize}

\section{Structure of this Thesis}
\label{s:introduction-structureofthisthesis}

This section is followed by the background \cref{c:background}, where the necessary theoretical background will be explained. 
In \cref{c:casestudy}, the setup of tweet collection is explained and the execution documented.
Afterwards, in \cref{c:analysis}, sentiments of collected tweets are determined and converted into a time series which is then compared to the time series of share prices.
Finally, in \cref{c:conclusion} the results of this work are summed up, and limitations and further points of interest are pointed out.


\chapter{Theoretical background}
    This section should provide the theoretical background and the foundation of the conducted study.
Therefore, this section is structured as follows: 
first, an introduction into option mining will be given in \cref{s:background-optionmining};
second, the market of social networks will be examined in \cref{s:background-socialnetworks};
and third, related work of stock market prediction will be presented in \cref{s:background-stockmarketprediction}.

\section{Option Mining} 
\label{s:background-optionmining}

Option mining is defined as extracting opinions from unstructured data using natural language processing techniques \cite[page 411]{Liu2007}.

According to Pang and Lee the two terms \emph{option mining} and \emph{sentiment analysis} are mere synonyms within this field of study \cite{Pang2008c}.
Therefore, this study will use the terms interchangeably.

There are several types of option mining available:

\begin{enumerate}
	\item 
	Sentiment classification: 
	This task is performed on a document level and classifies the text as positive or negative. 
	No further analysis is made what people may like or dislike 
	\cite[page 411]{Liu2007}.
	
	\item 
	Feature-based opinion mining and summarization: 
	This task dives deep into the text and analyzes the sentences on it's own.
	Furthermore, it discovers the opinions on objects which are liked or disliked.
	An object may be a product, service, topic or organization. 
	For example in an product review it detects the product features which have been described 
	\cite[page 412]{Liu2007}.
	
	\item
	Comparative sentence and relation mining:
	In this type of task product features are compared to the same or similar feature of another product.
	For example comparing two cameras: "the battery life of camera A is much shorter than that of camera B" 
	\cite[page 412]{Liu2007}. 
\end{enumerate}

This study will focus on short documents with given keywords in it. 
Therefore, we assume that the documents describing our targeted topic.
As a result the study will focus on sentiment classification.

Sentiment classification has some similarities with topic-based text classification, which classifies the topic of documents into predefined topic classes, for example sports, science or politics.
In topic based classification topic words are important, while they are unimportant for sentiment classification \cite[page 412f]{Liu2007}.

\subsection{Algorithms}
\label{ss:background-optionmining-alogrithms}

\citeauthor{Pang2002} performed a study which analyzed several machine learning algorithms and their suitability with sentiment classification.
These three algorithms have shown a good performance in text categorization studies and therefore they tried to apply these algorithms to sentiment classification \cite{Pang2002}.

\begin{table}
	\begin{center}
		\begin{tabular}{l l}
			\textbf{Element} & \textbf{Description} \\ \hline
			$c$ & Spefific document class \\
			$C$ & Document classes \\
			$d$ & Specific document \\
			$D$ & Documents \\
			$f_i$ & Specific feature \\
			$m$ & Count of features \\
			$n_i(d)$ & Count of features $i$ in document $d$ \\
			$P(x)$ & Probability of x \\
		\end{tabular}
	
		\label{tab:background-optionmining-algorithms-variables}
		\caption{Used variables by the algorithms}		
	\end{center}
\end{table}

\begin{description}
	\item[Naive Bayes.] 
    The \ac{NB} algorithm calculates the probabilities that a document $d$ belong to a class $c$.
	Furthermore, it implies that all classes are independent to each other, which does not hold true very often in a real world scenario.
	In sentiment classification tasks there are two classes available: positive and negative.
	Therefore the text is classified as class $c$ where $c* = arg max_c P(c | d)$.
	The general Bayes classifier can be seen in \autoref{eq:background-optionmining-algorithms-bayes} \cite{Pang2002}.
	
	\begin{equation}
		P_{NB}(c|d) = \frac{P(c) (\prod_{i=1}^{m} P(f_i|c)^{n_i(d)}) }{P(d)}
		\label{eq:background-optionmining-algorithms-bayes}
	\end{equation}
	
	Despite the fact that the Naive Bayes algorithm is very simple and the assumption of independent classes does not hold true in a real world scenario it performs surprisingly well \cite{Pang2002}.
	
	\item[Maximum Entropy Classification.]
  The \ac{ME} Classification succeeded in a number of language processing applications.
  The exponential form of the Maximum Entropy Classifier can be seen in \autoref{eq:background-optionmining-algorithms-maximumentropy} \cite{Pang2002}.
  
  \begin{equation}
    P_{ME}(c|d) = \frac{1}{Z(d)} exp \left( \sum_i^m \lambda_{i,c}F_{i,c}(d,c) \right)
    \label{eq:background-optionmining-algorithms-maximumentropy}
  \end{equation}
  
  Whereas $Z(d)$ is a normalization function as depicted in \autoref{eq:background-optionmining-algorithms-maximumentropy_Zd} \cite{Nigam1999} and $F_{i,c}(d,c)$ is a feature/class function for feature $f_i$ and class $c$ which is defined in \autoref{eq:background-optionmining-algorithms-maximumentropy_fic}.

  \begin{equation}
    Z(d) = \sum_c exp(\sum_i \lambda_{i,c} F_{i,c}(d,c))
      \label{eq:background-optionmining-algorithms-maximumentropy_Zd}
  \end{equation}

  \begin{equation}
  F_{i,c}(d,c') = 
    \begin{cases}
      1, & n_i(d) > 0 \text{ and } c' = c \\
      0  & \text{otherwise}
    \end{cases}
	\label{eq:background-optionmining-algorithms-maximumentropy_fic}
  \end{equation}

  $\lambda_{i,c}$ are feature weight parameters. A larger value of that parameter means that feature $f_i$ is considered as strong indicator for class $c$.
  The values for $\lambda_{i,c}$ are set in a way that the entropy of the training data set is maximized \cite{Pang2002}.
  	
	\item[Support Vector Machine.]
   \ac{SVM} are large-margin classifiers in contrast to the probabilistic Naive Bayes and Maximum Entropy.
   That means that in the two-category case (positive or negative) the training procedure looking for a hyperplane which maximizes the margin between the two groups \cite{Pang2002}.
   This scenario is depicted in \autoref{fig:background-optionmining-algorithms-svm} \cite[p. 275]{Cortes1995}.
      
   \begin{figure}[ht]
    \centering
    \includegraphics[width=.7\textwidth]{images/svm.png}
    \caption{An example of a two-category case, from \cite[p. 275]{Cortes1995}}
    \label{fig:background-optionmining-algorithms-svm}
  \end{figure}
	
\end{description}

%algorithms
%SVM
%naive bayesian

%bag of words 
%unigrams
%bigrams
%trigrams

%pre processing
%stop word
%stemming

\section{Social networks/microblogging services}
\label{s:background-socialnetworks}

In the last years many social networks started and many of them disappeared again.
There are several types of social networks available today: some of them are built on videos others on pictures and some of them mostly on text.
Those which are built on text are more easy to analyze and searchable for researchers worldwide.
But there are some constants in this field: Twitter, to identify one of them.
Twitter has become a valuable source of opinions, data and information for various researches \cite{Barbosa2010}.

Messages on Twitter are called tweets and are limited in length (140 for tweets before November 7th 2017; 280 characters since then, at least for english tweets \cite{Rosen2017TweetingEasier}) users have to concentrate on a specific topic precisely.
Therefore, Twitter is the perfect source of the public opinion as users discussing anything on Twitter \cite{Pagolu2016a}.

As a result the fields in which researches have used data from Twitter ranging from public opinions for politicans and polls (see \cite{Oconnor2010a,Patodkar2016a}) to the stock market and the prediction of stock prices and other factors (see \cite{Bollen2011a,Mittal2012a,Nguyen2015a,Pagolu2016a,Zhang2011a}).

% On the other hand keywords like cloud, machine learning and artificial intelligence are everywhere. 
% Amazon, Facebook, Google, IBM and Microsoft and other big players of the industry are making massive progress in these fields.

\section{Stock Market Prediction} 
\label{s:background-stockmarketprediction}

\citeauthor{Bollen2011a} noted that many researches assumed that the stock market is based on the random walk theory and the \ac{EMH}.
First, the \ac{EMH} suggests that the price of a security is reflected by all information available \cite{fama1965behavior, schumaker2009textual}.
Furthermore, the \ac{EMH} can be split into three forms: the Weak, the Semi-Strong and the Strong.

\begin{itemize}
    \item
        In Weak \ac{EMH} only historical information is incorporated in the current price \cite{schumaker2009textual}.

    \item
        In Semi-Strong \ac{EMH} uses historical and also currently available public information in the current price \cite{schumaker2009textual}.

    \item
        In Strong \ac{EMH} the Semi-Strong model is enhanced with currently available private information such as insider information in the share price \cite{schumaker2009textual}.
\end{itemize}

Therefore, the \ac{EMH} assumes that stock market prices are driven by new information such as news and will less depend on the current price or historical prices.
As news are unpredictable the stock market prices will follow a random walk pattern \cite{Bollen2011a}.

But \citeauthor{Bollen2011a} stated also that stock prices aren't following the random walk pattern completely and can therefore predicted in some way.
They stated that information available online may act as early indicators for changes.
This include the Google search queries which provide an early indicator for disease infections (see \url{https://www.google.org/flutrends/}) \cite{Bollen2011a}.


\chapter{Experimental Setup}
    \section{Introduction}
Some meaningless text to show how citation works \cite{latexcompanion}.

Cum vitae nulla scelerisque nulla erat ullamcorper id id at placerat volutpat hac vestibulum interdum posuere imperdiet a parturient ullamcorper risus sit montes. Amet ut himenaeos dapibus volutpat feugiat luctus consequat magna netus eu mi congue accumsan a dui a pulvinar. Porta a cubilia nisi arcu a class egestas fermentum consectetur magna venenatis et ullamcorper dolor habitant malesuada porta scelerisque ac sed platea a. Blandit mus mi imperdiet augue condimentum fames blandit feugiat eros nisi blandit scelerisque ut cum imperdiet auctor adipiscing pretium adipiscing vel consequat curae primis malesuada dignissim. Penatibus consequat mi ullamcorper platea senectus leo suspendisse a nulla nibh gravida scelerisque ac convallis platea neque at vestibulum vel. 

Elit nisi in nec consequat placerat euismod per mus nullam aliquam a in at ad lobortis netus tellus pharetra sociis ante adipiscing aliquam a. Gravida parturient nullam blandit scelerisque at dapibus eleifend convallis ad dolor natoque eget turpis rutrum non parturient in nibh aenean. Molestie adipiscing platea odio curabitur ullamcorper nisi ad dui vitae parturient ornare turpis odio eu vitae condimentum porta morbi vestibulum pharetra nullam. Tempor a quisque vestibulum at pretium ullamcorper vestibulum habitasse inceptos nascetur lacus sodales bibendum montes imperdiet ullamcorper a lorem eu netus. Leo adipiscing hendrerit leo scelerisque lectus laoreet vivamus a a mi placerat aliquam cursus nisi in a parturient consectetur vestibulum proin sagittis a consectetur elementum. 

Condimentum parturient nisl nullam ullamcorper vel ullamcorper sagittis parturient a dis a commodo nisl habitasse parturient proin. Mus suspendisse amet aliquet ornare ut elementum parturient eu lorem sodales fames vestibulum eu consectetur duis consectetur. Nunc ad suspendisse sagittis per quam lobortis hac fringilla proin ullamcorper vestibulum ullamcorper in torquent ac suspendisse sagittis sodales parturient torquent arcu inceptos penatibus. Aliquet ante sem ante gravida venenatis a suspendisse metus sem quis porttitor a parturient convallis vestibulum condimentum sed. Suspendisse condimentum nunc conubia duis a venenatis scelerisque ad vehicula mi habitant nec ad senectus bibendum malesuada a natoque a fringilla. Ultricies massa a ac penatibus nec a porttitor parturient consequat leo erat luctus magnis suscipit vestibulum a. 

\subsection{Random text}
A himenaeos turpis consectetur a platea id velit nisl dui justo fringilla semper tristique parturient. Vehicula ante aliquam quisque at suspendisse habitant a facilisi interdum sociosqu a feugiat aenean habitasse nostra ut. A arcu tristique duis parturient suspendisse donec mi ac mus venenatis in ullamcorper tristique massa lacus eget parturient torquent est rutrum vitae a. 

Vestibulum parturient litora pharetra cursus a duis adipiscing pulvinar at a pulvinar sed a condimentum per erat adipiscing ullamcorper condimentum augue ad euismod aenean enim dignissim dictum. A leo lacus tellus elit non vulputate hac a massa condimentum sed dis platea scelerisque. Parturient pharetra morbi quis vestibulum maecenas penatibus hac id nam leo consectetur turpis maecenas adipiscing suspendisse a lacus leo tincidunt parturient eu accumsan phasellus curae. Est habitasse adipiscing est ridiculus porttitor nostra at faucibus vestibulum luctus lacus vestibulum nisl cursus duis. Ullamcorper blandit platea fringilla et ullamcorper suspendisse dignissim in vel adipiscing parturient nisi volutpat a arcu curabitur eget tempor natoque nec rhoncus a. Fames fringilla suspendisse ac leo suscipit condimentum a a sed pulvinar scelerisque nostra adipiscing egestas. 

\subsubsection{Random squared text}
Fermentum a habitasse dis accumsan a elit praesent cursus id iaculis ad suspendisse aptent fusce a condimentum nunc commodo metus porta elit congue sceleris a ligula curabitur amet consequat. Ultrices curae est ad elit a venenatis adipiscing sodales vel quam eu mi cubilia a donec at a ultricies inceptos lacus euismod. In parturient parturient scelerisque parturient ac enim in vestibulum proin torquent a blandit augue a nisi suspendisse. Sodales convallis a cras eu iaculis mi commodo conubia adipiscing dictum adipiscing enim nunc quam parturient leo suspendisse adipiscing tristique habitasse ut sociosqu a lacinia. 

Hac potenti etiam porta vestibulum consectetur mus feugiat a parturient cum mi adipiscing parturient duis. A ipsum inceptos ad mollis nibh mauris parturient convallis augue parturient et malesuada a eros augue interdum sed proin himenaeos egestas ante ipsum id magna adipiscing a a. Ligula tellus consectetur a a ultricies class condimentum justo in rutrum orci nascetur sociosqu augue tempor habitant in. Fusce integer adipiscing integer scelerisque aliquet consectetur nam a magna ullamcorper a primis orci hac aenean lobortis a donec ac aptent nunc dapibus class adipiscing pulvinar nulla duis imperdiet. Arcu bibendum ac vehicula mollis condimentum ullamcorper mauris ac maecenas tempus a porta condimentum maecenas suscipit volutpat est tincidunt libero ut laoreet posuere dis. Mus ultricies pharetra suspendisse sem tempus interdum volutpat a ornare id et placerat venenatis rutrum ullamcorper habitant placerat vestibulum in.

\chapter{Conclusion}
    In a non-fiction book, a Conclusion is an ending section which states the concluding ideas and concepts of the preceding writing. This generally follows the body or perhaps a Afterword, and the conclusion may be followed by an Epilogue, Outro, Postscript, Appendix/Addendum, Glossary, Bibliography, Index, Errata, or a Colophon.

A common English usage misconception is that a paragraph has three to five sentences; single-word paragraphs can be seen in some professional writing, and journalists often use single-sentence paragraphs.

The crafting of clear, coherent paragraphs is the subject of considerable stylistic debate. Forms generally vary among types of writing. For example, newspapers, scientific journals, and fictional essays have somewhat different conventions for the placement of paragraph breaks.

English students are sometimes taught that a paragraph should have a topic sentence or "main idea", preferably first, and multiple "supporting" or "detail" sentences which explain or supply evidence. One technique of this type, intended for essay writing, is known as the Schaffer paragraph. For example, the following excerpt from Dr. Samuel Johnson's Lives of the English Poets, the first sentence is the main idea: that Joseph Addison is a skilled "describer of life and manners". The succeeding sentences are details that support and explain the main idea in a specific way.


% Future work:
% Issues with DMI TCAT (no interpolation!)
% sarcasm in tweets
% retweets?
% classification in positive, neutral and negative?
    
        %%------------------------END MATTER----------------------------%%
        %%--------------------------------------------------------------%%

\backmatter
        
\printbibliography[title={References}]
%\addcontentsline{toc}{chapter}{References}

\appendix
\chapter{Appendix A: Sentiment Analysis Python Script}
%\lstinputlisting[language=Python, firstline=37, lastline=45]{source_filename.py}
\lstinputlisting[language=Python,caption={Python script for sentiment analysis of tweets},label=lst:appendix-tweeets_analyzer]{./code-files/tweets_analyzer.py}

% \chapter{List of installed software at FH Wiener Neustadt}
% \includepdf[pages={1,2}]{externaldocs/itsoftware}

\end{document}