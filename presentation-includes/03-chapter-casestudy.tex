%!TEX root = ../presentation.tex

\section{Case Study}

\begin{frame}
  \frametitle{Companies \& Keywords}

  \begin{itemize}
    \item Determined five companies
      \begin{itemize}
        \item \ford{}
        \item \gm{}
        \item \hyundai{}
        \item \toyota{}
        \item \vw{}
      \end{itemize}

    \item Determined 23 keywords/brands
  \end{itemize}
\end{frame}

\note[itemize]{
  \item Table 3.1: Keywords/Brands, page 13-15 (ranking)
  \begin{itemize}
    \item Ford (5)
      \begin{itemize}
        \item Ford, Lincoln
      \end{itemize}

    \item GM (4)
      \begin{itemize}
        \item Baojun, Buick, Cadillac, Chevrolet, GMC, Holden, Jiefang, Wuling
      \end{itemize}

    \item Hyundai (3)
      \begin{itemize}
        \item Hyundai, KIA
      \end{itemize}

    \item Toyota (1)
      \begin{itemize}
        \item Daihatsu, Lexus, Toyota
      \end{itemize}

    \item VW (2)
      \begin{itemize}
        \item Audi, Bentley, Bugatti, Lamborghini, Porsche, Seat, Škoda, Volkswagen
      \end{itemize}

  \end{itemize}
  \item Table 3.2: Five Companies, page 15
}

\begin{frame}
  \frametitle{Stock Symbols}

  {\footnotesize
  \begin{table}
      \centering
      \begin{tabular}[c]{!l ^r ^l ^l}
        \hline
        \rowstyle{\bfseries}
          Company & \#cars\citep{OICA2016} & Market & Symbol  \\ \hline
          \ford{} & 6,429,485 & New York & F  \\
          \gm{} & 7,793,066 & New York & GM \\
          \hyundai{} & 7,889,538 & Korea & 005380.KS \\
          \toyota{} & 10,213,486 & Tokyo & 7203.T \\
          \vw{} & 10,126,281 & Frankfurt & VOW.F \\  \hline
        \end{tabular}
    \end{table}
  }
\end{frame}

\note[itemize]{
  \item Same as Table 3.2: Five Companies, page 15
  \item OICA = Organisation Internationale des Constructeurs d'Automobiles (engl. International Organization of Motor Vehicle Manufacturers)
  \item Daten von dem OICA corresponds survey 2016
}

\begin{frame}
  \frametitle{Gather Data}

  \begin{itemize}
    \item Tweets
      \begin{itemize}
        \item Using DMI-TCAT using Twitter Streaming API
        \item Gathered Tweets from \printdate{2018-02-28} and \printdate{2018-09-07}
      \end{itemize}
    
    \item Share Prices
      \begin{itemize}
        \item Yahoo Finance
      \end{itemize}
    
  \end{itemize}

\end{frame}

\note[itemize]{
  \item Why DMI-TCAT?: page 16/17
  \item Description of datasets beginning with page 24
}

\begin{frame}
  \frametitle{Tweets}

  {\footnotesize
  \begin{table}
      \centering
      \begin{tabular}{!l ^r ^r}
        \hline
        \rowstyle{\bfseries}
            Company     & \# captured tweets  & \# English tweets \\ \hline
            \ford{}     & \num{4518198}       & \num{3745447}     \\
            \gm{}       & \num{575547}        & \num{413817}      \\
            \hyundai{}  & \num{1895306}       & \num{697221}      \\
            \toyota{}   & \num{915868}        & \num{488913}      \\
            \vw{}       & \num{8244083}       & \num{6219350}     \\ \hline
            Total       & \num{16149002}      & \num{11565283}    \\ \hline
      \end{tabular}
    \end{table}
  }
    
\end{frame}